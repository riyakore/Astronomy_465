%\documentclass{article}
%\documentclass[preprint2]{aastex63}
%\usepackage{graphicx} % Required for inserting images


\documentclass[linenumbers,twocolumn]{aastex631}
\usepackage[utf8]{inputenc}
\usepackage{natbib}
\setcitestyle{round}
\usepackage{amsmath,amssymb}
\usepackage{afterpage}
\usepackage{xcolor}
\usepackage[caption=false]{subfig}
\usepackage{soul}
\usepackage{enumitem}
\usepackage{xcolor}
\usepackage{graphicx}
\usepackage[caption=false]{subfig}

\newcommand{\hi}{{\rm H\,{\small I}}}
\newcommand{\kms}{\ensuremath{{\rm km~s^{-1}}}}
\newcommand{\persc}{\ensuremath{{\rm cm^{-2}}}}
\newcommand{\msun}{\ensuremath{M_{\odot}}}

\begin{document}

\title{Astron 465 Optical Lab Report - CCD Characterization}

\author[0000-0002-3418-7817]{Riya Kore}
\affiliation{University of Wisconsin--Madison, Department of Astronomy, 475 N Charter St, Madison, WI 53703, USA}



\section{Introduction}

Charge-Coupled Devices (CCDs) are pivotal tools in modern optical astronomy, known for their exceptional sensitivity and dynamic range, making them ideal for capturing precise measurements of faint celestial sources. These detectors play a crucial role in various astronomical tasks, including photometry, astrometry, and spectroscopy. However, CCDs are not without their limitations—they introduce noise and exhibit non-uniform sensitivity, which can significantly affect the accuracy of the data collected. To ensure high-quality scientific measurements, it is essential to perform proper calibration of the CCD. This involves characterizing key detector properties such as gain and read noise, which serve as indicators of the CCD's efficiency and noise performance. \\

The primary goal of this lab was to understand the process of CCD calibration by developing and applying a standard reduction procedure. Specifically, this experiment focused on characterizing the CCD used in the Sterling Rooftop Observatory through the analysis of bias frames and flat-field images. Bias frames capture the inherent electronic noise in the CCD, while flat-field images are used to correct for pixel-to-pixel variations in sensitivity. By analyzing these calibration frames, critical parameters like gain—the conversion factor between the number of electrons generated by incident photons and the corresponding digital counts—and read noise—the electronic noise introduced during readout—were determined. These parameters are crucial for ensuring that the CCD delivers accurate and reliable measurements, particularly in high-precision photometry where precise noise control directly affects the quality of astronomical observations. \\

CCDs operate by converting incoming photons into electrical charges through the photoelectric effect. When light strikes the silicon surface of a CCD, electron-hole pairs are generated, accumulating in potential wells that correspond to individual pixels. During readout, these accumulated charges are shifted through a series of registers to an analog-to-digital converter, where they are converted into digital counts. This process is highly efficient, but it introduces several types of noise, including thermal noise (dark current), readout noise, and photon shot noise. Each of these noise sources can obscure faint signals from astronomical objects, making it essential to understand and mitigate them through proper calibration.

\subsection{Importance of Calibration in CCD Imaging}

Calibration is the process of correcting raw CCD data to ensure that the measurements accurately represent the true intensity of observed celestial objects. Without calibration, images may be affected by artifacts such as uneven sensitivity across pixels, electronic offsets, and thermal noise, making it challenging to extract meaningful information. To address these issues, this experiment employed two primary calibration techniques: bias subtraction and flat-field correction.
\begin{enumerate}
    \item Bias Frames: Bias frames are taken with zero exposure time and no incident light, capturing the intrinsic electronic noise of the CCD when it is read out. These frames serve as a baseline reference for electronic noise and are subtracted from all other images to remove this inherent noise component. The subtraction of bias frames ensures that any measured signal in subsequent images is free from contamination by the CCD's baseline electronic noise.
    \item Flat-field Images: Flat-field images are captured under uniform illumination using a diffuse light source, such as a twilight sky or a dome flat. These images are essential for correcting variations in pixel sensitivity across the CCD array, which can arise due to imperfections in the detector or the optical system. By dividing science images by a normalized flat-field image, the pixel-to-pixel variations are corrected, resulting in a more uniform response across the detector.
\end{enumerate}

Through the careful acquisition and analysis of these frames, the experiment aimed to derive the gain and read noise of the CCD, providing insights into its performance under observational conditions. Understanding these properties is vital for astronomers, as they influence the accuracy of measurements such as stellar magnitudes, galaxy brightness profiles, and other faint astronomical phenomena. Moreover, accurate characterization of the CCD ensures that observations are not compromised by instrumental artifacts, enabling reliable data for scientific research. \\

This report details the entire process of data acquisition, reduction, and analysis, from connecting to the CCD system at the Sterling Rooftop Observatory to processing the images and calculating key performance metrics. The results provide a thorough characterization of the CCD's behavior and offer recommendations for optimizing future observations to minimize noise and enhance data quality. These insights are particularly valuable as observational astronomy continues to demand increasingly precise measurements for advancing our understanding of the universe.

\section{Observations} \label{sec:observations}

The process of calibrating and characterizing the CCD camera involved the careful acquisition of bias and flat-field images using the telescope at the Sterling Rooftop Observatory. Each type of image served a specific purpose in the calibration process, and their acquisition required meticulous control over the equipment and conditions. The entire process was structured to ensure high-quality data collection while addressing any challenges that arose during the observations.

\subsection{Telescope Connection and Setup}
Data acquisition was conducted using the Apogee Alta F8300 CCD camera, controlled remotely from a laptop connected to the telescope system. The procedure ensured that all necessary steps were followed to maintain a stable connection and accurate control over the CCD's operational parameters. Table 1 shows the technical specs of the CCD used.

\begin{table}[h]
    \centering
    \begin{tabular}{c|c}
        Pixel Size & $5.4 \mu m \times 5.4 \mu m$ \\
        Array Size & $3326 \times 2504 \text{pixels}$ \\
        Sensor Area & $18 \text{mm} \times 13.5 \text{mm}$ \\
        Well Depth & 40,000 electrons per pixel \\
        Dark Current & $0.016 e^{-} \text{per pixel per second}$ \\
        Linearity & Better than 99\% \\
    \end{tabular}
    \caption{Technical specs of the CCD used}
    \label{tab:my_label}
\end{table}

\begin{enumerate}
    \item Establishing the Connection: The laptop was connected to the Raspberry Pi controller via a web browser, using the address \texttt{http://astroberry.astro.wisc.edu/desktop/}. This connection allowed access to the Astroberry interface, enabling remote control over the CCD camera and telescope.
    \item Accessing the Control Software: After logging in with the necessary credentials, the CCDciel software was launched, serving as the primary interface for managing the CCD's functions and controlling the telescope's positioning.
    \item Configuring CCD Settings: Within CCDciel, the connection between the CCD camera and other instruments was established by selecting the “Connect” option. This connection enabled adjustments to crucial settings such as exposure time, filter selection, and image type (e.g., "Bias" for bias frames or "Flat" for flat-field images).
    \item Monitoring the Cooling System: The CCD's cooling system was activated and monitored throughout the observation session. Proper cooling is vital to minimize thermal noise in the images, maintaining a consistent operating temperature of around -15°C. Stability in temperature was crucial to ensure that the bias and flat-field images accurately reflected the CCD’s behavior without being affected by temperature-induced noise variations.
\end{enumerate}
The connection setup was executed with precision to ensure minimal disruptions during data collection, providing the necessary control over the CCD and telescope systems.

\subsection{Bias Images}

Bias images are particularly important for calibrating science images and flat fields, as they account for any electronic offsets in the CCD's readout. A total of 20 individual bias frames were acquired during the observation session. These images were taken with the telescope's shutter closed and a zero exposure time, ensuring that no external light reached the detector. Each bias frame contained the baseline electronic noise of the CCD, represented as pixel-to-pixel variations in signal levels.
Figure~\ref{fig:bias_images} shows the Bias images obtained.

\begin{figure}[h]
    \centering
    \includegraphics[width=0.7\linewidth]{bias_img_2.png}
    \hfill
    \includegraphics[width=0.7\linewidth]{bias_img_3.png}
    \caption{Bias frames showing the intrinsic electronic noise of the CCD.}
    \label{fig:bias_images}
\end{figure}

\subsection{Flat-Field Images with Varying Exposure Times}
Flat-field images were obtained to correct for pixel-to-pixel sensitivity variations in the CCD, ensuring a uniform response across the detector. By illuminating the CCD with a uniform light source, these images highlight sensitivity differences among pixels. Multiple pairs of flat-field images were captured using the r-band filter, each with different exposure times (1s, 2s, and 10s). The range of exposure times aimed to cover the CCD’s dynamic range, from low-intensity signals to levels approaching saturation. \\

Two images were taken for each exposure time to calculate average intensity values and assess noise levels through difference imaging. This approach allowed for a more precise calibration of the CCD’s response at various illumination levels, ensuring accurate correction of the scientific images. \\

Figure~\ref{fig:flat_fields} show examples of flat-field images taken at different exposure times. The 1-second and 2-second exposure images highlight the sensitivity variations without any noticeable saturation, while the 10-second exposure reaches a higher signal level. All images were inspected to ensure they remained within a safe range, avoiding overexposure.

\begin{figure}[h]
    \centering
    \includegraphics[width=0.7\linewidth]{flat_1_img_1.png}
    \hfill
    \includegraphics[width=0.7\linewidth]{flat_2_img_1.png}
    \vskip\baselineskip
    \includegraphics[width=0.7\linewidth]{flat_10_img_1.png}
    \caption{Flat-field images at different exposure times showing pixel sensitivity variations. (Top: 1s exposure, Middle: 2s exposure, Bottom: 10s exposure)}
    \label{fig:flat_fields}
\end{figure}


\subsection{Flat-Field Images for Different Bands}
To fully characterize the CCD, additional flat-field images were taken using different filters (g, r, and i) at a consistent 2-second exposure time. These images help distinguish variations in sensitivity and throughput among the filters, which is critical for multi-wavelength observations. The g-band captures bluer light, while the r and i-bands cover progressively redder parts of the spectrum. Comparing flat-field images across these bands enables the identification of wavelength-dependent pixel response variations and detector behavior. \\

The flat-field images were carefully monitored to ensure no saturation, and adjustments were made to maintain uniform illumination. Figure~\ref{fig:g_band}, Figure~\ref{fig:r_band}, and Figure~\ref{fig:i_band} present flat-field images for each filter, showing the differences in response due to filter transmission characteristics.

\subsection{Challenges Encountered}
Several challenges were encountered during the data acquisition process, necessitating adjustments to the setup and methodology:
\begin{enumerate}
    \item Detector Cooling Issues: Maintaining a stable temperature in the CCD was a significant challenge, as the cooling system occasionally caused the detector to become too cold. This resulted in the formation of small ice crystals on the CCD surface, leading to unwanted artifacts in some of the flat-field images. These artifacts compromised the uniformity of the flat fields, requiring additional time and effort to adjust the cooling system and retake the affected images.
    \item Exposure Balance in Flat-Fields: Careful management of exposure times was crucial for obtaining high-quality flat-field images. Overexposed images risked saturating pixel wells, leading to loss of detail, while underexposed images did not provide sufficient signal for accurate calibration. (For example, 30s exposure time overexposed the images and the data taken was not fit to use). Monitoring the exposure levels closely and making iterative adjustments allowed for the acquisition of flat fields that balanced signal strength and linearity.
    \item Detector Defects in Bias Frames: Analysis of the bias frames revealed some structural anomalies, such as column defects and banding near the edges of the frames. These are typical features of CCDs that can introduce errors into the data if not properly accounted for. By averaging multiple bias frames, the effects of these anomalies were minimized, allowing for a more accurate subtraction of the bias signal from subsequent images.
\end{enumerate}

Despite these challenges, the final set of bias and flat-field images provided a robust foundation for characterizing the CCD and applying accurate calibrations to the scientific data.


\section{Data analysis} 
\label{sec:results}

This section outlines the procedures followed for characterizing the CCD's gain and read noise, utilizing bias frames and flat-field images as calibration tools. The analysis was performed using Python in a Jupyter Notebook environment, leveraging the \texttt{astropy.io.fits} module for reading FITS files, and \texttt{numpy} for calculations. Calculations that required averaging pixel values across all images utilized the parameter \texttt{axis=0} to ensure that the computations were performed on a per-pixel basis across the entire set of images. The analysis is divided into several key steps:

\subsection{Deriving the Plate Scale and Field of View}
Using the telescope's specifications, the plate scale, which converts pixel measurements to angular measurements in arcseconds, was calculated using the formula:
{\small
\begin{equation}
\text{Plate Scale(arcsec per pix)} = \frac{\text{Pix Size(in mm)}}{\text{Focal Len(in mm)}} \times 206265
\end{equation}
}

The plate scale is essential for determining the field of view of the CCD and allows for precise mapping of celestial coordinates to pixel coordinates. The plate scale value for this particular telescope and CCD is $0.27846$ arcseconds and the calculated field of view is $926 \times 697$.

\subsection{Determining Read Noise}
Read noise, which represents the inherent electronic noise of the CCD during readout, was calculated using pairs of bias images. The standard deviation of the difference between two bias frames provides a measure of the read noise:
\begin{equation}
\sigma_{\text{readnoise}} = \sqrt{\langle (B_1 - B_2)^2 \rangle}
\end{equation}
This calculation used \texttt{axis=0} to determine the standard deviation across all pixels in the image set, ensuring that the noise level was consistent for each pixel over multiple frames. For the 20 bias images taken, the read noise came out to be $\sigma_{\text{readnoise}} = 9.9048$.

\subsection{Analysis of Flat-Field Images}

Flat-field images were used to correct for pixel-to-pixel sensitivity variations in the CCD. The flat-field images were processed to calculate their mean intensity levels and the noise characteristics of the difference images. 
\begin{equation}
\sigma_{\Delta F} = \text{std}(F_1 - F_2)
\end{equation}
For each set of flat-field pairs with identical filters and exposure times, the noise was computed as the standard deviation of the difference between the two images, using \texttt{axis=0} for averaging across all pixels (shown in equation 3).

\subsection{Calculating Gain}

The gain of the CCD, which represents the conversion factor between charge (in electrons) and the digital counts (ADU), was derived using pairs of flat-field and bias images. For each pair of images, the gain was calculated with the following formula:
\begin{equation}
g = \frac{\left( \overline{F_1 + F_2} \right) - \left( \overline{B_1 + B_2} \right)}{\left( \sigma^2_{\Delta F} - \sigma^2_{\Delta B} \right)} \left[ \frac{e^-}{\text{ADU}} \right]
\end{equation}
In this equation:
\begin{itemize}
    \item \( \overline{F_1 + F_2} \) represents the average of two flat-field images.
    \item \( \overline{B_1 + B_2} \) is the average of two bias images.
    \item \( \sigma^2_{\Delta F} \) and \( \sigma^2_{\Delta B} \) are the variances of the differences between the flat-field and bias pairs, respectively.
\end{itemize}

The calculation used \texttt{axis=0} to compute the mean and standard deviation of each pixel across the set of images, ensuring that the values represented pixel-wise averages over the dataset.

In theory, the ideal gain value should be 1 \( \left[ \frac{e^-}{\text{ADU}} \right] \), meaning that each electron would correspond to one Analog-to-Digital Unit (ADU). However, the gain values obtained for different exposure times in this analysis were slightly different from this ideal. Specifically:
\begin{itemize}
    \item For a 1-second exposure: \( g_1 = 0.9769 \, \left[ \frac{e^-}{\text{ADU}} \right] \)
    \item For a 2-second exposure: \( g_2 = 0.9836 \, \left[ \frac{e^-}{\text{ADU}} \right] \)
    \item For a 10-second exposure: \( g_{10} = 1.0570 \, \left[ \frac{e^-}{\text{ADU}} \right] \)
\end{itemize}

These values indicate a close approximation to the theoretical gain of 1, with slight variations likely due to differences in exposure levels and the intrinsic characteristics of the CCD.

\subsection{Comparing Theoretical and Actual Gain}
Theoretical calculations for the gain were compared to the actual measured values, using the relationship between the observed standard deviation and the mean image intensity. The key difference between the two equations used is that CCDs do not directly measure the number of detected electrons (\(I[e^{-}]\)), but rather provide the counts in terms of Analog-to-Digital Units (ADUs). This relationship is expressed as:
\begin{equation}
    \sigma(I[e^{-}]) = \sqrt{I + \sigma^{2}_{\text{readnoise}[e^{-}]}}
\end{equation}

In this equation, \(\sigma(I[e^{-}])\) represents the standard deviation of the measured electron count, \(I\) is the mean intensity in electrons, and \(\sigma_{\text{readnoise}[e^{-}]}\) is the read noise in units of electrons. Since the CCD directly returns data in ADUs, the equation must be adapted to account for the gain (\(g\)), which converts ADUs to electron counts:
\begin{equation}
\sigma(I[\text{ADU}]) = g^{-1} \times \sqrt{I_{\text{ADU}} \times g + \sigma_{\text{readnoise,ADU}}^2}
\end{equation}

Here, \(\sigma(I[\text{ADU}])\) represents the standard deviation of the signal in ADUs, \(I_{\text{ADU}}\) is the mean intensity measured in ADUs, and \(\sigma_{\text{readnoise,ADU}}\) is the read noise in ADUs. \\

The transformation from equation (5) to equation (6) is essential because it enables a comparison between the theoretical gain-based model and the actual measurements derived from the CCD in terms of ADUs. The gain (\(g\)) serves as the scaling factor that translates the intensity and noise from ADUs to the corresponding electron counts. \\

To verify the accuracy of the derived gain, a log-log plot was generated comparing the observed standard deviation (\(\sigma\)) as a function of mean intensity (\(I\)). The theoretical model predicts a linear relationship between \(\log(\sigma)\) and \(\log(I)\) with a slope of 0.5 at higher intensity levels, where the shot noise (which scales as \(\sqrt{I}\)) dominates. At lower intensity levels, the plot flattens as the read noise (\(\sigma_{\text{readnoise,ADU}}\)) becomes more significant. \\

The comparison between the empirical data points and the theoretical curve is shown in Figure~\ref{fig:theoretical_vs_actual}. In the plot, the red line represents the theoretical curve derived from the gain calculation, while the blue points depict the measured standard deviations from the flat-field images. The alignment of the blue points with the red line across different intensity levels validates the gain values derived from the experiment, suggesting that the CCD behaves as expected according to the theoretical model.

\begin{figure}[h]
    \centering
    \includegraphics[width=1.0\linewidth]{optical_lab_1_theoretical_vs_actual.png}
    \caption{Comparison between the theoretical and actual standard deviation vs. mean image intensity. The red line shows the theoretical relationship, while the blue points represent the actual measurements.}
    \label{fig:theoretical_vs_actual}
\end{figure}

\subsection{Identification of Hot and Cold Pixels}
Hot and cold pixels were identified based on deviations in sensitivity from the mean, defined as pixels whose sensitivity deviated by more than 20\% from the average. Using \texttt{numpy masks} and \texttt{axis=0} for processing across all images, the number of such pixels was computed to evaluate the uniformity of the CCD’s response. For each exposure time (1s, 2s, and 10s), the following results were obtained:
\begin{itemize}
    \item For 1s exposure time: 37 hot pixels and 1 cold pixel.
    \item For 2s exposure time: 40 hot pixels and 1 cold pixel.
    \item For 10s exposure time: 33 hot pixels and 2 cold pixels.
\end{itemize}

These results indicate that the CCD exhibits a stable response over various exposure times, with a few isolated pixels deviating significantly from the average. These hot and cold pixels are a common characteristic in CCD detectors and can be corrected during data reduction.

\subsection{Flat-Field Comparisons Across Bandpasses}

Flat-field images were analyzed for the g, r, and i bands to assess pixel sensitivity variations across different wavelengths. Figures \ref{fig:g_band}, \ref{fig:r_band}, and \ref{fig:i_band} display the flat-field images, which show slight variations in sensitivity, likely due to differences in filter transmission and detector response. \\

The g-band, which corresponds to shorter wavelengths, exhibited a relatively uniform distribution of sensitivity across the detector, with minimal visible artifacts. However, due to the higher energy of shorter-wavelength photons, the g-band flat-field tends to highlight any minor variations in detector sensitivity, particularly near the edges of the field. This can be attributed to slight differences in filter uniformity and the interaction of blue light with the detector's sensitivity. \\

In the r-band flat-field, there are noticeable donut-shaped artifacts that appear consistently in the same positions as those seen in the g and i bands. These artifacts are indicative of dust particles on the CCD detector surface, as the fixed positions of these patterns across different filters suggest that the contamination is on the detector rather than in the optical path between the telescope and detector. The r-band sensitivity profile is smoother compared to the g-band due to the longer wavelength and lower energy of the red photons, but the dust contamination is more apparent. \\

The i-band flat-field image, which corresponds to the longer-wavelength, near-infrared region, reveals similar dust artifacts to those in the r-band but with a slightly different intensity pattern. The overall response in the i-band is influenced by the detector's decreasing sensitivity at longer wavelengths, resulting in a more pronounced central brightening in the flat-field. This central enhancement is often due to the vignetting effect, where the sensitivity decreases towards the edges of the field of view. \\

This comprehensive analysis provides a deeper understanding of the performance characteristics of the Apogee Alta F8300 CCD, facilitating more accurate calibration for future astronomical observations.

\begin{figure}[h]
    \centering
    \includegraphics[width=0.6\linewidth]{g_band_img.png}
    \caption{Flat-field image for the g-band showing pixel sensitivity variations.}
    \label{fig:g_band}
\end{figure}

\begin{figure}[h]
    \centering
    \includegraphics[width=0.6\linewidth]{r_band_img.png}
    \caption{Flat-field image for the r-band showing pixel sensitivity variations.}
    \label{fig:r_band}
\end{figure}

\begin{figure}[h]
    \centering
    \includegraphics[width=0.6\linewidth]{i_band_img.png}
    \caption{Flat-field image for the i-band showing pixel sensitivity variations.}
    \label{fig:i_band}
\end{figure}


\section{Conclusions}

This lab focused on characterizing the CCD detector used in the Sterling Rooftop Observatory by determining its gain, read noise, and evaluating the flat-field images. These characterizations are crucial for ensuring the accuracy of astronomical observations, as they provide insights into the intrinsic noise and calibration requirements of the CCD.

\subsection{Key Findings}

The gain of the CCD measures the conversion efficiency between electron counts and Analog-to-Digital Units (ADUs). A theoretical gain value of 1 $e^{-}/\text{ADU}$ represents an ideal conversion efficiency. In this experiment, gain values derived from flat-field pairs with exposure times of 1s, 2s, and 10s were found to be 0.9769, 0.9836, and 1.0570 $e^{-}/\text{ADU}$, respectively. These values closely align with the theoretical expectation, indicating that the detector's performance in terms of conversion efficiency is nearly optimal. Additionally, the read noise, which represents the inherent electronic noise present during the readout process, was measured at 9.9048 ADUs using the standard deviation of the bias frames. Understanding this read noise is crucial for characterizing the noise floor, especially in low-light observations where noise levels can significantly impact data quality. \\

The flat-field analysis involved taking images in the g, r, and i bands to assess pixel sensitivity variations across the CCD detector. Utilizing multiple filters with the same exposure time allowed for the detection of any wavelength-dependent variations in sensitivity or contamination, which could potentially impact subsequent photometric measurements. Analysis of the flat-field images revealed stationary donut-shaped artifacts, indicative of dust particles on the detector's surface rather than between the telescope and detector (If the donut shaped artifacts changed their position between images, it would indicate that the dust particles are not on the detector's surface). Furthermore, a starfish pattern was observed at the center of the flat-field images for shorter exposure times, likely due to insufficient time for the CCD or telescope to fully adjust to the target illumination, making this effect more prominent at shorter integration times. \\

The analysis of hot and cold pixels, defined as those deviating by more than 20\% from the mean sensitivity, provided additional insights into the detector's uniformity. For 1s, 2s, and 10s exposure times, there were 37, 40, and 33 hot pixels, respectively, while cold pixels remained limited to 1 for the 1s and 2s exposures and increased slightly to 2 for the 10s exposure. The relatively low number of cold pixels suggests that the CCD detector maintains a consistent sensitivity across its surface, while the presence of hot pixels indicates areas with heightened sensitivity, possibly due to minor imperfections or localized variations in the detector.

\subsection{Importance of Accurate Calibration}
The results of this experiment emphasizes the importance of accurate calibration in CCD-based astronomical observations. Bias frame subtraction is crucial for eliminating baseline noise, while flat-field corrections address pixel-to-pixel sensitivity variations, ensuring that the detected light reflects the true characteristics of astronomical objects. Understanding the gain allows for precise conversion from ADUs to electron counts, which is essential for the quantitative analysis of celestial sources. Proper calibration procedures help in reducing systematic errors, thus enhancing the reliability of scientific data derived from observations.

\subsection{Recommendations for Future Work}

To improve the accuracy and reliability of future observations, several measures can be recommended. Firstly, regular cleaning of the CCD window should be implemented to address the dust artifacts observed in the flat-field images, potentially enhancing the uniformity of collected data. Secondly, expanding the calibration sets, particularly increasing the number of bias and flat-field images, could help to minimize statistical uncertainties in gain and read noise measurements. Additionally, exploring the temperature dependencies of read noise and dark current could provide a deeper understanding of the CCD's behavior under varying conditions, leading to further refinement of calibration procedures.

\subsection{Conclusion}

In summary, this lab demonstrated a systematic approach to characterizing the performance of a CCD detector. By quantifying its gain, read noise, and sensitivity variations, the groundwork was laid for conducting accurate photometric and spectroscopic measurements in future observations. A thorough understanding and correction of these parameters is critical for achieving high precision in astronomical data acquisition, ensuring that the detected signals reflect the true nature of celestial objects. Such meticulous calibration enables researchers to derive meaningful scientific conclusions from their observational data, enhancing the overall quality of astronomical research.

\begin{thebibliography}{}

\bibitem[Howell(2019)]{Howell2019}
Howell, S. B. 2019, \textit{Handbook of CCD Astronomy}, Cambridge University Press

\bibitem[Janesick(2001)]{Janesick2001}
Janesick, J. R. 2001, \textit{Scientific Charge-Coupled Devices}, SPIE Press

\bibitem[Sterling(2023)]{Sterling2023}
Sterling Rooftop Observatory Manual, University of Wisconsin--Madison, 2023

\bibitem[Astro465(2024)]{Astro465}
Astro465 Class Slides and Lab Manual, University of Wisconsin--Madison, 2024

\end{thebibliography}


\end{document}

